
\newsection{ Pressure/tension control } \label{sec:npt}

Changes in the periodic boundary dimension are controlled using a vector of parameters $\vek{\alpha}$ that scales the coordinates from the original PBCs:
\begin{equation}
\vek{r}' = \{ r_x \alpha_x, r_y \alpha_y, r_z \alpha_z \}
\end{equation} 
Naturally the periodic cell dimensions are changed by the same values of $\alpha$.

Changing the value of $\alpha_x$ changes the shape of the underlying mesh, as well as the positions of all attached particles.
Not only is the potential energy changed, but the kinetic energy as well.

The kinetic energy $T$ is computed as
\begin{equation}
T = \frac{1}{2} \sum_i p_i \dot{q}_i
\end{equation}
where $p_i$ and $q_i$ are the conjugate momentum and coordinate of degree-of-freedom $i$, which might be a single Cartesian (e.g. $x$) or surface coordinate (e.g. $u$).
The quantity $\dot{q}_i$ is computed as $\hat{M}^{-1} \cdot \vek{p}$.
The matrix $\hat{M}$ is the same for each Cartesian dimension.
It is computed initially with $\vek{\alpha}=\{1,1,1\}$, scales as $\alpha^2$, as it is the outer product of $\frac{\vek{\partial r}}{\partial q_i}$ with itself.
Changing $\vek{\alpha}$ thus changes $T$ for the system and so $T$ must be considered when computing the probability $p$ of a Monte Carlo move attempt in $\alpha$:
\begin{equation}
p = \exp{-\beta (V_\textrm{new}-V_\textrm{old}+T_\textrm{new}-T_\textrm{old})}
\end{equation} 
where $\beta$ is the inverse temperature, $V$ is the potential energy, and subscripts label the quantities before and after the attempted move. 
